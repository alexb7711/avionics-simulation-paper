\section{GMSE Systems Architecture}

The objective of this section is to decompose the components of the simulation program from a
high-level description and interconnectivity of its components to low-level technical constructs of
essential components of GMSE. As simulation programs are no simple task to implement, multiple
components emerge to address specific tasks. At a systems level, these modules must then interact in
such a fashion to achieve the overall goal: avionics simulation.

An effective top-down method of presenting the taxonomy of architectures is as described in
\cite{levis_c4isr_2000}. The work breaks systems down into three architectures: operational
architecture, a systems architecture, and a technical architecture. The operational architecture
describes what the system does by describing how each of the packages perform and how they are
integrated together, the systems architecture states the major subsystems and how they behave, and
the technical architecture provides the details of each subsystem. In the following sections the
operational, systems, and technical architectures are presented in
\autoref{sec:operational-architecture}, \autoref{sec:systems-architecture}, and
\autoref{sec:technical-architecture}, respectively, for the GMSE. The primary focus of this work is
to deconstruct GMSE, the other components that interact with GMSE provide essential insight into the
constructs of GMSE. As such, description where required for external modules will be provided.

\subsection{Operational Architecture}
\label{sec:operational-architecture}

When describing a large piece of software, one can often decompose the software into a set of
packages. Packages describes software that are composed of multiple modules which ideally are
designed to perform a specialized task\footnote{This process can be extended ad nauseam to further
and further decompose items until its most primitive forms are derived. That will be avoided as much
as possible to keep the discussion relevant to the critical components.}. At the subsystem level,
there are three main packages: Core Avionics\footnote{Often times modules associated with core
avionics are also referred to as subsystems}, External Sensors and Modules, and the GMSE. The
modules and their interconnectivity is shown in \autoref{fig:operational-architecture}.

The core avionics package describes the suite of software being developed that that has immediate
agency to the behavior of the avionics system. For example, the Avionics Data Computer (ADC) is the
"brain" of the avionics system. It contains the vehicle's perceived external state as well as its
internal operating state. It also supplies flow of data at the correct rate to the correct system.
The Flight Control Module (FCM) utilizes the data from the ADC to assist in maintaining stability
and controlling the vehicle to the waypoints described by the planning algorithms utilized by the
Flight Navigation Module (FNC).

For the core avionics to function in the real world, a sense of the vehicle's immediate surroundings
must be established via various sensors and external modules. In particular, the sensors (more often
than not) are off-the-shelf products that require integration with the core avionics. The sensors in
this package may contain software modules either that mimic the behavior of a sensor, or include the
physical sensor in-the-loop of the simulation. This software package also contains "external
modules" which are other off-the-shelf components that require integration into the core avionics.
As an example, if one desires to incorporate data linking to transfer data in real-time to a base
station, the physical device to transfer the data via radio frequency (RF) would be included as an
"external module".

The objective of the GMSE package is to drive both the sensor modules and core avionics by spawning
all the necessary processes, executing tasks at their scheduled times, calculating the ``state of
the world'', calculating the state of ``ownship'', as well as data logging and plotting. That is,
the objective of GMSE is to provide all the ``services'' that would otherwise have been provided by
real-world scenarios. The means of connectivity between each of the packages is provided by the
Small Dynamic Subsystem Interface Protocol (SDSIP). A description of the communications is provided
in \autoref{sec:sdsip}.

\begin{figure*}[ht]
\includegraphics[width=\textwidth]{operational-architecture}
\caption{}
\label{fig:operational-architecture}
\end{figure*}

The GMSE package consists of several modules\footnote{Probably better described as packages as well,
but for the sake of description these components will be referred to as modules.} as depicted in
\autoref{fig:operational-architecture}. These modules are the GMSE driver, log, and physics engine.
The driver module is the part of the GMSE package that initializes the simulation, spawns all the
required processes, and ensures the correct modeled external modules and sensors are updated at the
correct times. The logging module of the GMSE package inputs both the avionics data and the raw data
created by the physics engine. The physics engine module is the back end that models the world
components such as the physical location of the aircraft, wind and other disturbances, as well as
obstacles.

\subsection{Systems Architecture}
\label{sec:systems-architecture}

\subsection{Technical Architecture} \label{sec:technical-architecture}

